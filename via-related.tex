\section{Related Work}
\label{sec:via:related}


\mypara{Overlay routing}
Overlay networking has been explored in a variety of contexts, such as virtual private networks (VPNs) and multicast~\cite{mbone,ALMI-USITS01,Multicast-Sigcomm02}. Of interest to us here is work focused on overlay routing with a view to improving performance~\cite{Detour-Sigcomm99,RON-SOSP01}. This work showed that performance in terms of network metrics such as delay and packet loss, and also reliability, could be improved by using an overlay path that traverses well-chosen waypoints. 

Despite this promise,  overlay routing for performance gains has not seen much adoption in practice, for several reasons including the last-mile performance bottlenecks encountered in using client nodes as peers and the policy issues involved in turning stub networks (e.g., university campus networks) into de facto transit networks. Perhaps most importantly, these efforts involved building up overlay networks from scratch, both in terms of physical infrastructure and network probing, which limited their scale.

Our work revisits the idea of overlay routing in the context of (a) global-scale managed networks, so the global infrastructure already exists and need not be built up from scratch, and (b) a large-scale interactive real-time service, \skype, which provides both a compelling need for improving performance and (passive) measurements to obviate the need for active network probing.


\mypara{Evolution of AV conferencing services} The architecture of audio-video conferencing services has been evolving, with a trend towards leveraging cloud resources. A case in point is Skype, which started off with a peer-to-peer approach to NAT and firewall traversal, with some well-connected clients with public IP addresses serving as super-nodes~\cite{Skype-GI08}. However, in the recent years, Skype has moved to a hybrid model~\cite{VideoTelephony-IMC12}, with some super-nodes hosted in the cloud~\cite{Skype-Zdnet13}. 
It has been reported that Google Hangouts uses relays in the cloud for all calls, and moreover also has streams traverse the cloud backbone from one relay to another~\cite{VideoTelephony-IMC12}. 

Our work is in line with these trends, but focused on performance rather than NAT/firewall traversal. Also, since we focus on managed networks, being selective in which streams are routed via the cloud is crucial in our context.

\mypara{CDN server selection} Optimal server selection is a much-studied problem, especially in the context of content distribution networks \cite{DONAR-Sigcomm10,youtubecdn}. The main considerations in the selection process are typically proximity of the client to replicas and the load on the replicas. The main distinction of our work is our focus on client-to-client communication, which means that relay selection needs to focus on end-to-end performance rather than just between the cloud edges and the client.

\mypara{Internet performance prediction} There is a large body of work on Internet performance prediction~\cite{IDMaps-ToN01,GNP-Infocom02,iplaneosdi}, with a focus on metrics such as bandwidth, delay, and packet loss rate. The general approach is to probe the network selectively, at chosen times and along chosen paths, and then to use the measurements to either embed the network nodes in a coordinate space~\cite{dabek2004vivaldi} or estimate the performance of network segments using network tomography techniques~\cite{Tomography-StatSci04}.
 Since we have access to network metrics for a large volume of calls, our work focuses on leveraging this data rather than performing active measurements.


\mypara{Measurement studies} Over the years, there have been a number of measurement studies of large Internet services, including web sites~\cite{Padmanabhan-Sigcomm00},  CDNs~\cite{CDN-OSDI02}, and video-on-demand streaming~\cite{Streaming-Usits01,sigcomm11}. There have also been studies of audio-video conferencing by working outside the system, say by running active measurements to Skype super-nodes~\cite{SkypeMeasurement-IPTPS07} or sniffing traffic in modest-size deployments~\cite{VideoTelephony-IMC12}. 
To our knowledge, this work is the first study of a commercial VoIP service at scale by directly working with end-to-end performance metrics recorded by the communicating peers themselves.

\mypara{Estimating VoIP Quality} Several models have been proposed and studied for estimating VoIP quality, typically the Mean Opinion Score (MOS), based on network performance metrics~\cite{cole,SkypeUserSatisfaction-Sigcomm06,SkypeMeasurement-IPTPS07,SkypeQuality-WMUST12,SkypeQoE-Multimedia12}. These models vary in the particular network metrics and codecs they consider. In Section~\ref{sec:via:potential}, we used the model proposed in ~\cite{cole}, which is based on the E-Model defined by the ITU~\cite{itu-e-model15}. 
