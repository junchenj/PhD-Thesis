\chapter{Introduction}


Today's Internet is an ``eyeball economy'' driven by applications, such 
as Internet video streaming (e.g., the share of video traffic of all consumer Internet 
traffic  hit 70\% in 2015 and is forecasted to reach 82\% of consumer Internet traffic by 
2020~\cite{cisco-forecast-2015}) and Internet telephony (e.g., Skype users spend over 
2 billion minutes talking to each other every day~\cite{skype-2-billion-minutes}). 
As these applications rely on user engagement to generate revenues, 
it has become of paramount importance that the
application providers ensure high user-perceived 
{\em Quality of Experience} ({\em QoE}) in order to maintain high user 
engagement~\cite{sigcomm13athula,akamai-imc12}.
For instance, recent research shows that even one short video buffering 
interruption can lead to 39\% less time spent watching online videos and 
cause substantial  revenue losses for ad-based video sites. 
Suboptimal QoE can negatively affect subscription-based 
service providers as well; e.g., our study with Microsoft Skype 
shows that most Skype users give low rating to calls when experiencing more than
1.2\% packet loss rate~\cite{via}.

Given the importance of QoE, understanding and improving the QoE of Internet applications 
have gained increasing attention from both academic and industry efforts. 
This trend is best illustrated by the recent growth in the number of 
publications (e.g.,~\cite{sigcomm13athula,sigcomm12,
wang2014speedy,sigcomm11,eona,akamai-imc12}), workshops 
(e.g.,~\cite{workshop-wmust,workshop-fhmn,workshop-qoe}), 
as well as commercial offerings (e.g.,~\cite{conviva,artizanetworks}) for optimizing the QoE 
of Internet video streaming, Internet telephony, mobile 
apps, and web services.

Despite the intense research towards better Internet QoE, 
recent measurement studies have shown that existing approaches have failed 
to deliver the QoE needed by today's applications. 
%Figure~\ref{fig:intro:badqoe} presents the QoE distribution and the 
%prevalence of bad QoE in video streaming and Internet telephony, 
%two of the most popular applications today.
%Figure~\ref{subfig:intro-badqoe-video-buffering} shows that among 
%over 300 million video sessions from 376 content 
%providers, 
For instance, our study based on over 300 million 
video sessions from 376 content providers showed that
%while most viewers did not experience noticeable re-buffering interruptions,
12\% video sessions spent over 1\% of time in re-buffering interruptions (video stalls),
and 5\% sessions waste even 10\% of the view time in 
re-buffering~\cite{jiang2013shedding}, which could significantly reduce user engagement, 
especially for live content~\cite{sigcomm11}. 
%Similarly, Figure~\ref{subfig:intro-badqoe-skype-lossrate} shows that 
%among 430 million Skype calls that were not relayed by an intermediate
Similar QoE problems are pervasive in other applications as well. 
Our study based on quality measurements of 430 million 
Skype calls showed that 17\% calls experienced over 
1.2\% packet loss in the call's duration~\cite{via}, which 
can cause frustrating user experience~\cite{itu,cisco-voip}.
%Note that the packet loss rate is the average value over the
%call's duration during which there may be transient spikes
%(e.g., loss burst) resulting in even worse experience.
(We elaborate on these quality problems in Chapter~\ref{ch:related}.)

%\begin{itemize}
%\item Overview of Internet applications: video streaming and internet telephony
%\item Re-architect the core -- high deployment cost
%\item Per-edge adaptation -- too narrow view
%\end{itemize}

%\section{Ensuring High QoE is Challenging}

\section{Fundamental Limitations of Prior Approaches}


%\begin{figure}[t!]
%\captionsetup[subfigure]{justification=centering,farskip=-1pt,captionskip=5pt}
%\centering
%%\hspace{-0.5cm}
%\subfloat[Video streaming]
%{
%        \includegraphics[width=0.35\textwidth]{figures/intro-badqoe-video-buffering.pdf}
%        \label{subfig:intro-badqoe-video-buffering}
%}
%%\hspace{-0.1cm}
%\subfloat[Internet telephony]
%{
%        \includegraphics[width=0.35\textwidth]{figures/intro-badqoe-skype-lossrate.pdf}
%        \label{subfig:intro-badqoe-skype-lossrate}
%}
%%\vspace{-0.2cm}
%\caption{QoE distributions of video streaming and Internet telephony. 
%The figures shows that a substantial fraction of video sessions (12\%) 
%and VoIP calls (17\%) suffer from bad QoE (over 1\% buffering ratio 
%and 1.2\% packet loss rate, respectively).
%Buffering ratio is the fraction of video session duration spent in 
%re-buffering (video stalls), and is one of the key metrics of video 
%streaming QoE. 
%Packet loss rate is calculated over the call's duration, and is shown to 
%have significant impact on VoIP user experience.}
%%\vspace{-0.1cm}
%\label{fig:intro:badqoe}
%\end{figure}

%\mypara{Limitations of prior approaches} 
These QoE problems stem from the fundamental limitations of prior 
approaches. There are two broadly defined classes of prior approaches,
%To understand these limitations, it is helpful to view the 
%prior approaches in two broadly defined classes, 
%with the key difference lying in
whose key difference lies in
{\em where to implement the functionality of QoE optimization}.

%To see why prior research has failed to deliver desirable 
%QoE in a substantial fraction of cases, one has to understand
%their fundamental limitations. 
%Prior networking approaches to Internet QoE optimization can be 
%classified to two broad categories depending on the answer to the key 
%architectural 
%question ``{\em where to implement the functionality of QoE optimization?}''


\begin{itemize}
\item First, the {\em in-network} approach seeks to improve QoE by improving
the quality of service (QoS) of ISPs and in-network services through better designs 
of in-network devices (e.g., routers, switches, and middleboxes) and routing schemes.
%seeks to re-architect in-network devices such as routers, switches
%and middleboxes, so that ISPs can provide better or even guaranteed 
%quality of service. 
Although the  in-network approach has inspired influential projects and 
enormous intellectual legacy (e.g.,~\cite{demers1989analysis,csfq,active-network,red,ecn,ccn}),
it is fundamentally limited by in-network devices' lack of visibility to user-perceived QoE,
and thus it cannot react to QoE problems that are not reflected by lower-level QoS metrics.
%since they have little access to client-side applications.
Moreover, it is difficult, and increasingly so, to make substantial changes 
to the network core, despite recent efforts to facilitate it~\cite{feamster2014road}.

\item Second, the {\em endpoint} approach seeks to improve QoE by using
intelligent logic running at individual endpoints to react to changes in
network conditions, in order to fully utilize the existing network resources.
This approach is pervasively used in application-level protocols (
e.g.,~\cite{dash,festive,butkiewicz2015klotski,diversifi,sprout})
and transport-level protocols (e.g.,~\cite{jacobson1988congestion,cubic,remy,pcc}).
Unlike the in-network approach, the endpoint approach has direct insight to user-perceived 
QoE and is arguably more deployable.
However, the endpoint approach is fundamentally limited by 
individual endpoints' local visibility to network conditions.
As a result, each endpoint reacts to the changes in network conditions and 
resource availability only after they have affected the QoE,
and when it reacts, it uses trial-and-error strategies driven by only local information.
Both aspects lead to suboptimal QoE when network conditions change constantly 
(e.g., flashcrowd) or when quality of the beginning of a session is critical (e.g., video streaming)
%can only react to fluctuating network conditions based 
%only on {\em local information} observed by individual endpoints could be suboptimal.
%For instance, it takes a video player several seconds (sometimes, even tens of seconds)
%to converge to an optimal combination of CDN and 
%bitrate~\cite{dda}.
%Moreover, it is untenable for such trial-and-error strategies driven by 
%single-endpoint information to find optimal decisions from the growing decision spaces
%in many applications. 
%we observe a growing decision space of potential control decisions 
%to optimize application quality.
%Therefore, it is untenable to use trial-and-error strategies driven by 
%single-session feedback to find optimal decisions in the growing decision space. 

%Due to this limited knowledge of network conditions, it is difficult for 
%the endpoint adaptation to cope with the {\em growing decision space} 
%of potential control decisions to optimize application quality.

%of endpoint adaptation suffer from two limitations: 
%(1) it can only use the information visible to individual endpoint to react to dynamic network 
%conditions or resource availability,
%and (2) it typically uses manually designed strategies. 
%These two features inherently mismatch two recent trends of Internet applications
%First, we observe a {\em growing decision
%space} of potential control decisions to optimize application quality.
%Consequently, trial-and-error strategies driven by single-session feedback are
%fundamentally inefficient and slow in exploring the decision space and reacting to
%changes. For instance, it takes a video player several chunks (roughly 10s of
%seconds) to converge to an optimal combination of CDN and bitrate~\cite{dda-report}.
%Second,  we see an {\em increasing heterogeneity} in operating conditions, each requiring
%different control logic and parameters.  For instance, TCP parameters such
%as initial congestion window and AIMD parameters could be tweaked to work
%better in different operating conditions~\cite{remy,googleinitwindow}.
\end{itemize}

In essence, both prior approaches have to compromise either
{\em visibility to user-perceived QoE} or {\em visibility to network conditions}, 
both of which are critical to achieving desirable QoE in practice. 
In contrast, this dissertation approaches the question of where to implement 
QoE optimization with a radically different answer, and demonstrates that the new approach 
can get the best of both worlds.


\section{New Paradigm: Data-Driven Networking}

This dissertation is inspired by a recent paradigm shift in networking research 
towards {\em Data-Driven Networking} ({\em \ddn}) -- network protocols should 
be driven not only by information available to one device, but by {\em real-time} 
information gathered from {\em multiple} devices~\cite{ddn-comsnet}.
Prior work has shown the potential of this paradigm shift in improving the 
QoE of Internet applications (e.g.,~\cite{sigcomm12,footprint}).
As illustrated in Figure~\ref{fig:intro:contrast}, 
the basic idea is that QoE could be substantially improved, if 
there is a centralized controller that maintains a global view of real-time 
network conditions by gathering QoE measured from many (typically, tens of 
thousands or more) application sessions\footnote{We use ``client'' to denote 
where a ``session'' is actually run.} and uses this global view to make  
optimal decisions regarding the adaptation of individual sessions.
%maintain a global view of 
%network conditions based on the QoE measured from many (typically, tens of 
%thousands or more) of application sessions\footnote{We use ``client'' to denote 
%where a ``session'' is actually run.} and use this real-time global view to make  
%optimal decisions regarding configurations or adaptations of individual sessions.

%A case in point is C3~\cite{c3}, where a video content provider 
%collects session-level QoE measurements (e.g., buffering events) from 
%the video players of its viewers, and predicts for any new session the best
%bitrate and CDN selection based on the QoE measurements of 
%history and concurrent video sessions.

%we can improve QoE by accurately predicting network conditions
%based on {\em a real-time, global view of QoE measured from millions of 
%end users~\cite{ddn-comsnet}.}


%\mypara{\ddn controller} 
%The key component in \ddn is a logically centralized 
%{\em controller}, which updates a real-time, 
%global view of network conditions by 
%QoE observed by applications running on 
%many end users in real time, and uses this global view of
%network conditions  to make  optimal decisions 
%regarding configurations or adaptations of clients or application 
%sessions\footnote{We use ``client'' to denote where a ``session'' 
%is actually run.}.


\begin{figure}[t!]
\captionsetup[subfigure]{justification=centering,farskip=-1pt,captionskip=5pt}
\centering
%\hspace{-0.5cm}
\subfloat[Classic approaches.]
{
        \includegraphics[width=0.4\textwidth]{figures/intro-classic.pdf}
        \label{subfig:problem-of-epsilon:change}
}
%\hspace{-0.1cm}
\subfloat[The new data-driven approach (\ddn).]
{
        \includegraphics[width=0.4\textwidth]{figures/intro-ddn.pdf}
        \label{subfig:problem-of-slow:load}
}
%\vspace{-0.2cm}
\caption{Contrasting \ddn paradigm with classic approaches. 
The key distinction lies in where to implement the functionality of QoE optimization 
(symbolized by the gears): prior approaches implement 
it in the in-network devices or individual endpoints, whereas \ddn implements 
it in the controller that maintains a real-time global view of QoE of millions of endpoints.}
\vspace{-0.1cm}
\label{fig:intro:contrast}
\end{figure}


In essence, \ddn offers a different answer to the key architectural question 
``where to implement the functionality of QoE optimization''.
Unlike the prior approaches which 
optimize QoE by in-network devices or individual endpoints, 
\ddn envisions a logically centralized controller that optimizes QoE based on 
a global view of real-time QoE of many endpoint.

%As shown in Figure~\ref{fig:intro:contrast}, \ddn represents a new approach
%to the architectural question: ``where to implement the functionality of QoE 
%optimization''.
%While the classic approaches implement QoE optimization in the network 
%core or individual endpoints, \ddn implements it in the controller 
%that maintains a real-time global view of QoE collected from
%client-side applications running in millions of end users.

\mypara{Advantages of \ddn} 
\ddn enjoys the benefits of both in-network and endpoint approaches, while avoiding
their pitfalls.
On one hand, \ddn retains the ethos of the endpoint approach that it can directly optimize
for user-perceived QoE.
On the other hand, moving the functionality of QoE optimization from endpoints to 
the logically centralized controller addresses individual endpoints' lacking of visibility to network 
conditions by gathering information from many endpoints, thus achieving 
the best of both worlds of in-network and endpoint approaches.

%%The key advantage of \ddn over prior approaches is that 
%%\ddn retains the ethos of the endpoint approach that it can directly optimize
%%for user-perceived QoE, while addressing its insufficient visibility to network 
%%conditions by gathering information from many endpoints, thus achieving 
%%the best of both worlds of in-network and endpoint approaches.
%Moving the functionality of QoE optimization to the centralized \ddn controller 
%%enjoys two key architectural advantages:
%enjoys the benefits of both in-network and endpoint approaches, while avoiding
%their pitfalls.
%\begin{enumerate}
%\item Unlike the in-network approach, \ddn can monitor client-side applications
%and thus can directly optimize user-perceived QoE, rather than indirect metrics.
%%as it has access to client-side applications,
%Moreover, it is more readily deployable than the in-network approach.
%\item Unlike the endpoint approach, \ddn compensates the lack of 
%visibility of network conditions at one endpoint by a real-time, global view of QoE 
%observed from many endpoints, thus addressing the key limitation of 
%the endpoint adaptation. 
%%Instead of reacting to dynamic network conditions with 
%%a local view, \ddn can {\em predict}  the QoE that a session would 
%%experience if it uses certain decision.
%\end{enumerate}

%\mypara{Technology trends} 
\mypara{``Technology pushes'' behind \ddn} 
%\vspace{0.2cm}
%\noindent{\bf Why now?}
While optimizing application quality by information of multiple sessions is not 
new, \ddn, unlike its precursors (e.g.,~\cite{spand}), is fortuitously aligned with several recent  technology pushes:
\begin{enumerate}
\item Many application providers today have widely deployed client-side 
instrumentations that can collect real-time in-situ QoE data 
en masse from clients (e.g.,~\cite{sigcomm11,via,akamai-imc12,artizanetworks}). 
\item The emergence of large-scale data analytics platforms and cloud infrastructure 
provides the ability to extract insights efficiently from large corpses 
of data (e.g.,~\cite{spark}) and streams of updates 
(e.g.,~\cite{zaharia2013discretized}). 
%Such ability enables optimal decision making based on
%real-time data-driven predictions~\cite{velox-cidr}.
\item Logically centralized control platforms are commonly deployed by many 
application providers (e.g., content providers~\cite{c3}, web services~\cite{footprint}) and CDN providers (e.g.,~\cite{chen2015end,mukerjee2015practical}).
\end{enumerate}

%\begin{itemize}
%\item {\em More measurement data in networking:}
%Many application providers today have widely deployed client-side 
%instrumentation to collect real-time performance data. 
%
%\item {\em ``Big data'' platforms finally a reality:}
%The emergence of large-scale analytics systems provides the ability to extract insights efficiently from large corpses 
%of data (e.g.,~\cite{spark}) and from stream of updates (e.g.,~\cite{dstream}). 
%Such ability enables optimal decision making based on real-time data-driven 
%predictions~\cite{velox}.
%
%\item {\em Wide use of control platforms:}
%Control plane platforms have been built by many individual subsystems (e.g., ISPs, video service providers and CDNs).
%
%\end{itemize}


%To take a concrete example, let’s think about Netflix player. 
%Traditionally, it takes a Netflix video player tens of seconds to converge to a good 
%quality, while in the new data-driven approach, Netflix can achieve much better 
%quality by using real-time quality measurement from many video players to directly 
%predict the best configuration for individual Netflix players.


%\subsection{Opportunities}

%\mypara{Opportunities of data-driven networking} 
%The data-driven approach is fortuitously aligned with several recent technology trends. Specifically, 
%
%\begin{itemize}
%\item {\em More measurement data in networking:}
%Many application providers today have widely deployed client-side 
%instrumentation to collect real-time performance data. \jc{examples}
%
%\item {\em ``Big data'' platforms finally a reality:}
%The emergence of large-scale analytics systems provides the ability to extract insights efficiently from large corpses 
%of data (e.g.,~\cite{spark}) and from stream of updates (e.g.,~\cite{dstream}). 
%Such ability enables optimal decision making based on real-time data-driven 
%predictions~\cite{velox}. \jc{examples}
%
%\item {\em Wide use of control platforms:}
%Control plane platforms have been built by many individual subsystems (e.g., ISPs, video service providers and CDNs). \jc{examples}
%
%\end{itemize}

\section{Making Data-Driven Networking Practical}

While prior work has shown that \ddn can potentially improve QoE, it is not 
clear how to fully realize this potential in practice.
{\em The main contribution of this dissertation is the first suite of solutions 
to make \ddn practical for QoE optimization}. In particular, 
%, and the demonstration that QoE of video streaming and Internet telephony can be 
%substantially improved by the application of \ddn.
%We demonstrate that QoE can be substantially improved by leveraging QoE 
%observed from millions of endpoints to maintain an up-to-date
%global view of network conditions.
%To achieve this, 
we identify key algorithmic and architectural challenges to apply \ddn to 
QoE optimization, address these challenges by novel algorithm and system 
designs that integrate machine-learning techniques with domain-specific  insights,
and use real-world deployment and large-scale 
emulation to demonstrate that our solutions can substantially improve 
the QoE of video streaming and Internet telephony.
%(We will elaborate them in more details in Chapter~\ref{ch:overview}.)

\subsection{Key Challenges}
%While \ddn enjoys several advantages over 
%classic approaches, 
From our experience of applying \ddn in various applications, 
two fundamental challenges emerge as key to unleashing \ddn's full potential.
%We observe two fundamental challenges 
%that are key to unleashing \ddn's full potential.
%We focus on two fundamental challenges unique to \ddn:

\begin{enumerate}

\item 
%{\em The need for expressive models} to capture complex factors affecting QoE. 
{\bf Expressive models:} 
\ddn needs to turn QoE measurements of millions of 
different application sessions into actionable insights.
%e.g., which sessions can be used to inform the decision of a new session.
%such as a predictive model to predict QoE based on session-level features, from 
%the data stream of QoE measurements. 
Thus, we need expressive models to capture the complex network-level and 
application-level factors that affect QoE. 

\item 
%{\em The need for scalable platforms} to make real-time decisions with fresh data from geo-distributed clients.
{\bf Scalable platforms:}
\ddn needs to turn the actionable insights into 
real-time control decisions to be performed by
geo-distributed clients. 
Thus, we need scalable platforms that can respond to geo-distributed 
clients in real time with decisions based on the most up-to-date insights 
extracted from information of other sessions.

\end{enumerate}


\begin{figure}[t!]
\centering
\includegraphics[width=0.8\textwidth]{figures/intro-contribution.pdf}
%\vspace{-0.3cm}
\caption{The main contribution of this dissertation is to present the first suite of 
solutions (bottom) to address the key challenges of \ddn (top) by leveraging the 
unifying insight (middle) that QoE-determining factors exhibit persistent structures.}
\label{fig:intro-contribution}
\end{figure}


\subsection{Unifying Insight}
We address these challenges by integrating 
machine learning techniques with
%As a result, our 
%solutions achieve better QoE 
%than using off-the-shelf machine learning solutions. 
the unifying insight that there are {\em persistent critical structures} in the relationship
between session-level features, decisions, and QoE 
(as illustrated in Figure~\ref{fig:intro-contribution}).
At a high level, these structures have two distinctive features:
%(1) they help identify network sessions with similar 
%QoE-determining factors, and (2) they tend to persist on timescales of tens of minutes.
(1) they allow us to build expressive models that can identify network 
sessions with similar QoE-determining factors, and 
(2) because these structures tend to be persistent, we can 
build scalable platforms by decoupling offline structure-learning 
processes and real-time decision making processes.
To see an intuitive example of a persistent critical structure, let us consider 
a group of video sessions bottlenecked by a congested link.
The quality of these sessions may vary over time, but the fact
that these video sessions are bottlenecked by the same congested link remains true 
for the whole duration of the congestion event.
In this example, the correlation among the quality of these sessions 
is a persistent critical structure, which manifests the underlying congestion.
(We will give a formal definition of persistent structures in
Section~\ref{sec:overview:unifying}.)


%\section{Contributions}



%\mypara{Challenges}
%While \ddn enjoys several advantages over 
%classic approaches, we observe two fundamental challenges 
%that are key to unleashing \ddn's full potential.
%%We focus on two fundamental challenges unique to \ddn:
%
%\begin{enumerate}
%
%\item 
%%{\em The need for expressive models} to capture complex factors affecting QoE. 
%First, \ddn needs to extract actionable insights from the QoE measurement data, 
%such as which sessions can be used to inform the decision of a new session.
%%such as a predictive model to predict QoE based on session-level features, from 
%%the data stream of QoE measurements. 
%In short, we need {\em expressive models} to capture the potentially complex 
%factors affecting QoE. 
%
%\item 
%%{\em The need for scalable platforms} to make real-time decisions with fresh data from geo-distributed clients.
%Second, \ddn needs to turn the actionable insights into 
%real-time control decisions to be performed by applications running in
%geo-distributed clients. 
%Therefore, we need {\em scalable platforms} that can respond to geo-distributed 
%clients in real time with decisions based on fresh data from other clients.
%
%\end{enumerate}

%\mypara{Key insight}
%In this dissertation, we address these challenges in practice by integrating 
%several domain-specific insights in networked applications with 
%standard machine learning algorithms and systems. 
%%As a result, our 
%%solutions achieve better QoE 
%%than using off-the-shelf machine learning solutions. 
%The unifying theme underlying these domain-specific insights is 
%that there are {\em persistent structures} in the relationship
%between session-level features, decisions, and QoE.
%These structures help identify network sessions with similar 
%QoE-determining factors, and that such structures tend to be 
%persistent on timescales of tens of minutes.
%To see an intuitive example of these persistent structures, let us consider 
%a group of video sessions bottlenecked by a congested link.
%The throughput of these flows may vary over time, but the fact
%that these video sessions are bottlenecked by this congested link remains true 
%for the whole duration of the congestion event.
%In this example,  the persistent correlation between these sessions' 
%network path and their QoE is a manifestation of the underlying congestion.
%We will give a formal definition of persistent structures in
%Section~\ref{sec:overview:unifying}.



%\begin{table}[]
%\centering
%\begin{tabular}{lll}
%\\ \hline
%{\bf Challenges} & {\bf Key ideas} & {\bf Published work}        \\ \hline\hline
%\begin{tabular}[c]{@{}l@{}}{\em Expressive models} for complex \\ QoE-determining factors\end{tabular}            & \begin{tabular}[c]{@{}l@{}}Use structures to identify sessions \\ with similar QoE-determining factors\end{tabular} & \begin{tabular}[c]{@{}l@{}}CFA~\cite{cfa}, VIA~\cite{via},  \\ CS2P~\cite{cs2p}\end{tabular}         \\ \hline
%\begin{tabular}[c]{@{}l@{}}{\em Scalable platforms} for real-time \\ decision making with fresh data\end{tabular} & \begin{tabular}[c]{@{}l@{}}Decouple offline structure learning \\ and the real-time decision making\end{tabular}    & \begin{tabular}[c]{@{}l@{}}Pytheas~\cite{pytheas}, CFA~\cite{cfa},  \\ C3~\cite{c3}, VDN~\cite{mukerjee2015practical}\end{tabular}\\ \hline
%\end{tabular}
%\caption{Summary of how the insight of persistent structures is used to address 
%the two key technical challenges of data-driven QoE optimization. The last 
%column shows the published work corresponding to these ideas.}
%\label{tab:contributions}
%\end{table}



\subsection{Proposed Solutions}
%\mypara{Proposed solutions}
%Table~\ref{tab:contributions} 
Inspired by the insight of persistent critical structures, this dissertation develops 
three concrete solutions
%of this dissertation.
to address the two aforementioned challenges in the context
of Internet video and Internet telephony. 
%Based on these ideas, this dissertation has developed novel algorithms 
%and end-to-end systems, and deployed them in production settings to 
%improve QoE for Internet video streaming
%and Internet telephony. 
Next, we briefly describe the three components that constitute 
this dissertation.



\begin{itemize}

\item {\bf Expressive QoE Prediction Using Critical Features} 
(Chapter~\ref{ch:cfa} and~\ref{ch:dda}). 
Prior work has shown a substantial room for improving video QoE by 
dynamically selecting the optimal CDN and bitrate for individual video 
sessions based on a real-time global view of network 
conditions~\cite{sigcomm12}.
To realize this promise, we have developed CFA~\cite{cfa} (a video QoE prediction 
system that can accurately predict the quality of a video client if it uses 
certain CDN and bitrate), and DDA~\cite{dda} (a throughput prediction system
to accurately predict end-to-end throughput at the beginning of a 
 video session to help determine the highest-yet-sustainable initial bitrate).
Both techniques are inspired by the domain-specific insight of 
{\em persistent critical features}, an instantiation of persistent critical structures, that
each video session has a small set of critical features that ultimately 
determines its video quality, and these critical features change much 
more slowly than video quality, and thus can be practically 
learned from history data.
%This insight enables us to learn complex prediction models from long-term historical data (thus expressing complex relations between video quality and session features), and update the models by short-term historical data in near real time (thus capturing quality fluctuation as well).
%The insight of persistent critical features turns out to be more general than video streaming; e.g., I have also applied the same insight to accurate prediction of TCP throughput~\cite{cs2p}, which leads to 11\% higher video bitrate than state-of-the-art adaptive bitrate players (e.g., Netflix players) with no extra buffering.

\item {\bf Group-Based Exploration-Exploitation at Scale} 
(Chapter~\ref{ch:pytheas}). 
While CFA and DDA show promising QoE improvement by formulating the 
data-driven QoE optimization as a prediction problem, this formulation is necessarily 
incomplete, as it suffers from a biased visibility and cannot respond to sudden changes.
Drawing on a parallel from machine learning, 
we argue that data-driven QoE optimization should instead be 
cast as a process of {\em real-time exploration-exploitation}. 
To scale the real-time exploration-exploitation process to millions of 
application 
sessions running in geo-distributed clients, we have developed 
a control platform called Pytheas~\cite{pytheas}, which 
%To apply this new abstraction to network applications at scale, 
%we develop a control platform called Pytheas~\cite{pytheas}. To scale
%to millions of application sessions running geo-distributed clients, Pytheas 
relies on another illustration of persistent critical structures that 
the sessions that exhibit similar QoE behaviors have similar network-level 
features (e.g., IP prefix), and thus their fresh data could be collected 
by the same geo-distributed front-end cluster close to the clients of
these sessions. Inspired by this insight, Pytheas uses a scheme called 
{\em group-based exploration-exploitation},  which decomposes the 
global exploration-exploitation process of all sessions into 
subprocesses, each managing a group of similar sessions and running in
the geo-distributed front-end cluster that has the fresh measurement data 
of these sessions.

%, each of which runs in a geo-distributed front-end cluster 
% and makes real-time decision for a group of similar sessions
%based on their fresh measurement data.

\item {\bf Tackling Large Decision Spaces via Guided Exploration} 
(Chapter~\ref{ch:via}).
The last project tackles a challenge of large decision spaces, which 
is particularly relevant in Internet telephony.
In the first large-scale study on VoIP\footnote{We use the terms Internet telephony 
and VoIP interchangeably.} quality, we found that 
%a substantial fraction of Skype calls suffer from poor network performance, 
%and that 
there is substantial room for improving Skype quality by routing each 
call through the optimal relay clusters in Microsoft's cloud.
However, identifying a close-to-optimal relay for each Skype call in practice 
is challenging, 
due to the sheer number of possible relay paths 
(in hundreds) and their dynamic performance (which could change 
on timescales of minutes). Neither prediction-based methods (e.g., 
CFA) nor those based on 
exploration and exploitation (e.g., Pytheas) would suffice to handle such a 
large decision space.
Our key insight to address this challenge is another manifestation of the 
persistent critical structures that, for each pair of caller 
AS and callee AS, there is a {\em small and stable subset} of relays that 
almost always contains the best relay path. 
%This insight has two implications: 
%(1) because this subset of relays is stable, it can be learned from history; and 
%(2) because this subset has only a few relays (less than five), it can be explored efficiently even with limited data.
We have developed {VIA}~\cite{via}, 
a Skype relay selection system that achieves close-to-optimal quality 
using the concept of {\em guided exploration},
which, instead of exploring the whole decision space of all possible relay choices, 
learns a small set of promising relays for each 
AS pair based on long-term (e.g., daily) historical data, and 
explores these promising relays using most calls in near real time.

\end{itemize}


%
%\jc{add why internet video, voip, and how to generalize}
\mypara{Generalizability beyond Internet video and telephony}
Finally, while this dissertation has mostly focused on 
Internet video and Internet telephony, 
we observe similar data-driven opportunities in other applications
(e.g., CDN overlay routing~\cite{mukerjee2015practical}, 
web proxy selection~\cite{footprint})
where solutions proposed in this dissertation are applicable.
%some of which I have personally involved as well.
%We also observe parallel efforts that hint towards the potential
%of using a similar data-driven approach in other applications, such
%as web services~\cite{footprint}.
Besides application-layer quality, the insight of persistent structures may be applied 
to optimize transport-layer performance.
For instance, DDA demonstrates the feasibility of predicting end-to-end 
throughput by using persistent structures to aggregate information of multiple network flows.
These observations indicate the potential {\em generalizability} of our solution to enable
unleash more data-driven opportunities in a broader set of scenarios in 
networking and distributed systems.
%for particular applications might be 
%{\em generalized} to other applications and potentially to solutions for
%a broader set of challenges in networking and distributed systems.

%\subsection{Evaluation}
%These improvements can lead to significant benefit for the application providers.





%\mypara{Deployment}

\section{Summary of Results}
%In order to evaluate the solutions proposed in this dissertation, 
%I focus on answering the following questions. 
In evaluating the solutions proposed in this dissertation, 
we focus on answering two questions.

\begin{itemize}

\item {\em How much can QoE be improved by the proposed solutions?}
The ultimate goal of \ddn is to improve QoE. 
%Rather than evaluating multiple solution together, 
We examine the 
contribution of each solution by evaluating the incremental QoE 
improvement of adding one component at a time. 
For instance, by predicting the best CDN and bitrate selections based on 
a global view of network conditions, CFA reduces video re-buffering time 
on average by 32\% compared to a state-of-the-art client-side logic using 
local information (Chapter~\ref{ch:cfa});
and by re-casting the data-driven QoE optimization as a real-time 
exploration-and-exploitation process, Pytheas further reduces the 
re-buffering time on average by 30\% over CFA (Chapter~\ref{ch:pytheas}).
In Chapter~\ref{ch:via}, we show that VIA achieves better VoIP QoE
than CFA and Pytheas
by addressing the new challenge of large decision spaces in
Internet telephony.
Moreover, this dissertation also tries to identify the circumstances under which
the proposed solutions achieve more QoE improvement.
For instance, in Chapter~\ref{ch:via}, we observe improvement of 
VIA on both international and domestic Skype calls, but international calls
have a higher magnitude of improvement than domestic ones.

%Chapter~\ref{ch:cfa} shows that by predicting the best 
%CDN and bitrate selections based on a real-time, global view of network 
%conditions, CFA reduces re-buffering time on average by 32\% compared 
%with those selections made by a state-of-the-art client-side logic using 
%local information.
%Then Chapter~\ref{ch:pytheas} shows that Pytheas can further reduce the 
%re-buffering time on average by 30\% over CFA by recasting the 
%data-driven QoE optimization as a 
%real-time exploration-and-exploitation process.
%Finally, in Chapter~\ref{ch:via}, we show that VIA achieves better QoE
%than CFA and Pytheas in the context of a VoIP service 
%by reducing large decision spaces unique to Internet telephony.


\item {\em Can the proposed solutions be deployed at real scale?}
Scalability is another critical metric to evaluate the proposed solutions. 
Any proposed solution must operate at the scale of a large application 
provider.
%(e.g., YouTube has billions of video sessions per
%day over the world~\cite{youtube-stats}).
%This means that the decisions must be made 
%based on history data at a magnitude of hundreds GB or more, and that
%the history data from geo-distributed 
%clients must be up-to-date to maintain an accurate view of network conditions.
In Chapter~\ref{ch:cfa}, we show that CFA can update video QoE 
prediction every tens of seconds with sub-second response 
time to the scale of a large video content provider (i.e., 10 million sessions every day), 
and in Chapter~\ref{ch:pytheas}, we show that Pytheas throughput 
scales horizontally with more machines in the controller, and that 
30 CloudLab instances can make decisions for the population of a site
like YouTube (5 billion sessions per day) with measurement data of 
concurrent sessions with less than a second of delay.

\end{itemize}


%\mypara{Evaluation methodology}
%\mypara{Dataset}
To demonstrate the benefit of our solutions in realistic settings, 
our evaluation methodology combines real-world pilot deployment and 
emulation/simulation driven by large-scale datasets collected from real  users.
%this dissertation relies on either real-world pilot deployment or emulation 
%driven by large-scale datasets collected from real application traffic.
For instance, in Chapter~\ref{ch:cfa}, we integrated CFA in a production 
system~\cite{c3} that provided video optimization service for major content
providers in the US.
%to select CDNs and bitrates for real users for one of the biggest
%content providers in the US. 
We deployed CFA on one of these content providers to improve QoE for
150,000 sessions each day. We performed A/B tests (where each algorithm
was used on a random subset of clients) to evaluate
the improvement of CFA over baseline random decision
makers, which many video optimization services use 
by default (modulo business arrangement like price).

%
%\mypara{Evaluation metrics} 
Finally, in order to evaluate application QoE in a reliable 
and scalable fashion, this dissertation does not use subjective QoE 
metrics (e.g., user-provided score), but focuses on the metrics that can be
objectively measured and known to have great impact on user satisfaction 
and engagement.
For instance, video QoE is measured by buffering time, start-up delay,
and average bitrate, each of which has been shown to have strong
correlation with user engagement in multiple 
studies~\cite{sigcomm11,akamai-imc12}.
While subjective metrics can directly reflect user satisfaction, we
choose to use these objectively measurable 
metrics as a proxy for real user satisfaction for two reasons:
(1) they can be passively collected en masse by instrumentation code 
running in client devices without any user input, and (2) they
are less noisy than subjective metrics which can be affected by factors
(e.g., content or personal preference) beyond the scope of this dissertation.

%The QoE metrics against which different systems are compared 
%are objectively measurable metrics
%depend on the particular application under consideration.



\section{Organization}
The rest of this dissertation is organized as follows.
Chapter~\ref{ch:related} begins with the background information of
Internet video and Internet telephony, including their QoE problems
today and current distribution infrastructures.
It then discusses two recent research directions closely 
related to this dissertation: 
quality optimization of Internet applications and application of data-driven 
techniques in networked systems.
In particular, it introduces a taxonomy of prior work on quality 
optimization, which emphasizes the trade-offs between 
more visibility of QoE feedback and more visibility of network
conditions.

Chapter~\ref{ch:overview} presents an overview of the main insight and 
ideas of this dissertation.
It begins with a formal description of the \ddn paradigm,
concrete example applications that can benefit from \ddn, and a
perspective on its advantages over prior approaches.
It then elaborates the key technical challenges in making \ddn practical. 
Finally, it describes our key insight of persistent structures in 
QoE-determining factors, and how the insight inspires our solutions to 
address the \ddn's challenges.
%It concludes with the contrast between the proposed solutions and classic 
%approaches, in order to elaborate their advantages and limitations.
Chapter~\ref{ch:measurement} presents a large-scale structural analysis
on the video and VoIP QoE problems in the wild.
It provides empirical evidence of the persistent structures in QoE-determining
factors. 

Chapters \ref{ch:cfa}, \ref{ch:dda}, \ref{ch:pytheas}, and \ref{ch:via} 
describe four important components of this dissertation: 
(1) CFA and DDA optimize video streaming quality by predicting 
video QoE and end-to-end throughput using a global and real-time view of 
network conditions;
(2) Pytheas optimizes quality of Internet-scale applications by re-casting
the \ddn process as a real-time exploration-exploitation process 
over millions of geo-distributed clients at scale; and
(3) VIA addresses the challenge of large decision spaces, and uses 
VoIP as a use-case where it optimizes network performance for VoIP calls 
by selecting the optimal relay clusters.

Chapter~\ref{ch:concl} summarizes the contributions of the dissertation, 
discusses the limitations of the proposed solutions, and ends with future 
work.










