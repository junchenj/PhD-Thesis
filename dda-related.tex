\section{Related Work}
\label{sec:dda:related}

At a high-level, our work is related to prior work in measuring Internet path
properties,   bandwidth measurements, and video-specific bitrate selection.
With respect to prior measurement work, our key contribution is showing a
practical data-driven approach for throughput prediction.  In terms of  video,
our predictive approach offers a more systematic bitrate selection mechanism.


\mypara{Measuring path properties} Studies on path properties have shown
prevalence and persistence of network bottlenecks
(e.g.,~\cite{hu2005measurement}), constancy of various network
metrics~\cite{zhang2001constancy}, longitudinal patterns of cellular
performance (e.g.,~\cite{nikravesh2014mobile}), and spatial similarity of
network performance (e.g.,~\cite{balakrishnan1997analyzing}). While \name is
inspired by these insights, it addresses a key gap because these efforts fall
short of providing a prescriptive algorithm for throughout prediction.

% the shown substantial throughput similarity, but as shown in
%\Section\ref{sec:analysis}, achieving accurate prediction is hard because of
%complex underlying interaction between sessions' throughput.

%%Existing approaches to
%prediction of bandwidth-related metrics (e.g., available bandwidth, throughput)
%%generally fall into three categories. The most similar approach to ours is
%shared prediction.

\mypara{Bandwidth measurement}  Unlike prior ``path mapping'' efforts
(e.g.,~\cite{iplaneosdi,spand,ramasubramanian2009treeness,
dabek2004vivaldi}), \name uses a data-driven model based on available session
features (e.g., ISP, device). Specifically,  video  measurements are taken
within a constraint sandbox environment (e.g., browser) that do not offer interface for path
information (e.g., traceroute).  Other approaches use  packet-level probing to
estimate the end-to-end performance metrics (e.g.,~\cite{prasad2003bandwidth,
hu2004locating, strauss2003measurement, jain2003end}).  Unlike \name, these
require additional measurement and often need full client/server-side control which
is often infeasible in the wild.  A third class of approaches leverages the
history of the  same client-server pair (e.g.,~\cite{vazhkudai2001predicting,
jain2005end, swany2002multivariate, mirza2007machine, he2005predictability}).
However, they are less reliable when the available history of the same client
and server is sparse. 

%Third approach, which is the most similar to ours, is to predict the
%performance between a host and server using the information between other
%hosts and servers. However, they either rely on network topology to correlate
%measurements (e.g., ~\cite{madhyastha2006iplane, seshan1997spand}), or fall
%short of predicting throughput (e.g.,~\cite{ramasubramanian2009treeness,
%dabek2004vivaldi}), while \name relies on only session features to
%automatically discover sesssion with similar throughput and achieve accurate
%prediction. 

%Similar to
%\name, predictive bitrate selection picks bitrate based on prediction of
%throughput. However,

\mypara{Bitrate selection} Choosing high and sustainable bitrate is
critical to video quality of experience~\cite{sigcomm13athula}.  
 Existing methods (e.g.,~\cite{miller2015low,festive})
require either history measurement between the same client and server or the
player to probe the server to predict the throughput. In contrast, \name is able to predict
throughput before a session starts. 
%In this perspective, video control plane
%(e.g.,~\cite{sigcomm12,c3}) is similar to \name, but it fails to provide
%details on how they correlate sessions with similar quality and falls short of
%predicting throughput directly.  
 Other approaches include switching bitrate midstream (e.g.,~\cite{huang2014buffer,tian2012towards,yin2014toward})
 but do not focus on the initial bitrate problem which is the focus of \name.

% But the
%buffer-based approach is not sufficient to simultaneously optimize for higher
%bitrate and startup delay~\cite{yin2014toward}. 


%\mypara{End-to-end property prediction}
%\begin{packeditemize}
%\item History-based predictor (\cite{vazhkudai2001predicting}, bandwidth variation range~\cite{jain2005end}, multivariate network weather service~\cite{swany2002multivariate}, \cite{mirza2007machine,he2005predictability})
%\item Spatial similarity of throughput is studied in \cite{seshan1997spand} and \cite{balakrishnan1997analyzing}.
%\item \cite{madhyastha2006iplane} periodically probes the network from vantage points.
%\item{Measurement tools for bandwidth estimation} 
%\begin{packeditemize}
%\item \cite{prasad2003bandwidth} overview of tools for various bandwidth-related metrics.
%\item \cite{hu2004locating} developes Pathneck for available bandwdith estimation
%\item \cite{strauss2003measurement} developes Spruce for available bandwidth estimation 
%\item \cite{jain2003end} develops Pathload
%\item \cite{wang2014timing} middlebox-based bandwidth estimation
%\end{packeditemize}
%\end{packeditemize}
%
%
%\mypara{Measurement studies on path properties}
%\begin{packeditemize}
%\item \cite{hu2005measurement} found bottleneck persistency, sharing of bottlenecks, 
%\item various steadiness properties~\cite{zhang2001constancy}
%\item \cite{nikravesh2014mobile} longtitudinal study on mobile end-to-end properties.
%\item \cite{luckie2014challenges} studies inter-domain congestion link
%\end{packeditemize}
%
%\mypara{Bitrate-adaptive streaming}
%\begin{packeditemize}
%\item \cite{miller2015low} studies short-term throughput fluctuation and presents a bitrate adaptation algorithm that maximizes QoE by throughput prediction.
%\item \cite{zou2015can}: how much improvement can be brought by accurate throughput prediction.
%\item \cite{yin2014toward}: accurate throughput prediction can enable QoE optimization
%\end{packeditemize}
%
%
%
