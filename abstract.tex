\begin{abstract}

The Internet is an eyeball economy dominated by applications such as 
Internet video and Internet telephony, whose revenue streams crucially 
depend on user-perceived Quality of Experience (QoE).
%For instance, even one short buffering interruption can lead to much less 
%video viewing time and cause significant revenue losses for the video site. 
Despite intense research towards better QoE, existing approaches
have failed to achieve the QoE needed by today's  applications,
because they are not acting on the ``right feedback signal'':
they either seek to rearchitect the in-network devices 
which have little information on user-perceived QoE, 
or propose end-to-end protocols which have limited knowledge on 
network conditions.
%Two classes of solutions have been proposed: in-network approach
%which re-architects the in-network devices to improve network quality of service, 
%and endpoint approach which uses local information of individual endpoints
%to react to dynamic network conditions.
%Their limitations stem from the fundamental trade-off between 
%visibility to {\em user-perceived QoE} and visibility to {\em network conditions}, 
%both of which are crucial to ensuring high QoE;
%%while endpoints have direct access to user-perceived QoE, they have limited 
%%knowledge of the network conditions.
%they either seek to rearchitect the in-network devices which have little visibility to QoE, 
%or propose end-to-end protocols which have limited visibility to 
%network conditions.

%{\em where they implement the 
%functionality of QoE optimization}. 
%they either require costly re-architecting of 
%the in-network devices (i.e., little visibility to QoE), or rely on suboptimal endpoint-based 
%protocols (i.e., limited knowledge of the network conditions).
%
%they have to sacrifice 
%visibility to user-perceived QoE or visibility to network conditions, both of which are 
%crucial to ensuring high QoE:
%they either require costly re-architecting of 
%the in-network devices (i.e., little visibility to QoE), or rely on suboptimal endpoint-based 
%protocols (i.e., limited knowledge of the network conditions).

The key contribution of this dissertation is to bridge the long-standing gap between 
the visibility to user-perceived QoE and the visibility to network conditions by a 
data-driven approach. 
%by demonstrating the 
%feasibility of a new approach, called data-driven networking, 
%inspired by the recent success of data-driven 
%techniques in other fields of computing.
The thesis is  that {\em one can substantially improve QoE by maintaining
a global view of up-to-date network conditions based on the QoE information
collected from many endpoints}.
In essence, this thesis revisits the architectural question of where to
implement the functionality of QoE optimization.
Unlike prior work which optimizes QoE by in-network devices or 
individual endpoints, our approach optimizes QoE by a logically centralized 
controller that retains the visibility to QoE while attaining a global view
of real-time network conditions by consolidating information from many endpoints.

To prove the thesis, this dissertation provides a suite of solutions to address 
two fundamental challenges.
First, we need expressive models to capture complex relations among 
QoE, decisions, and application sessions who share similar QoE-determining 
factors.
Second, we need scalable platforms to make real-time decisions with fresh data 
from geo-distributed clients. 

Our key insight that there are {\em persistent critical structures} in the relations between 
QoE and session-level features. 
These structures allow us to build expressive models that can identify network
sessions with similar QoE-determining factors, and their temporal persistence allows
us to build scalable platforms by decoupling offline structure-learning
processes and real-time decision making processes.
We have developed algorithms and end-to-end systems, which integrate 
machine-learning techniques with our insight of persistent
critical structures. We have used real-world deployment and 
simulation driven by real datasets to show that our solutions 
can yield substantial QoE improvement and consequently higher user 
engagement for video streaming and Internet telephony 
than existing solutions as well as many standard machine 
learning solutions. 

%Our techniques address 
%these challenges by integrating machine learning algorithms 
%and systems with several domain-specific insights in networked 
%applications. The key insight is that there are 
%{\em persistent structures} 
%in the relations between QoE and session-level features.
%Inspired by this insight, this dissertation has developed algorithms 
%and end-to-end systems, and used real-world deployment and 
%simulation driven by real datasets to show that our solutions 
%can yield substantial QoE improvement and consequently higher user 
%engagement for video streaming and Internet telephony 
%than today's techniques as well as many standard machine 
%learning solutions. 

\end{abstract}
